\section{The structure of the thesis}

There should be 5--8 numbered chapters in a thesis, including
Introduction and Conclusion.  If necessary, you can use sections
and subsections to give the thesis a more fine-grained structure.

The chapters that lie between Introduction and Conclusion are 
collectively called the \textit{body} of the thesis.  It is
often said to start with a \textit{background}, which is then
followed either by a \textit{main theorem}, a \textit{constructive part}
or an \textit{empirical part}.

Instead of simply listing headings of different levels it is recommended to let every
heading be followed by at least a short passage of text.

Headings should reflect the content of the coming part (chapter, section).

\subsection{Background}
\label{background}

The goal of the background part of a thesis is to develop the
theoretical background required in the thesis.  The idea is that a
reader of the thesis should be able to understand all the special 
concepts and methods used in the thesis.
A good thesis also gives well-argued reasons for why exactly these
concepts and methods are in use in the thesis (with the main
alternatives given in the literature mentioned).

The best way to present and use of the theoretical background depends on
what the thesis is about. The background part of a
fully mathematical work differs considerably from the background part of a quantitative or qualitative
empirical study. 

Some suggestions:
\begin{itemize}
\item In a fully mathematical (theoretical) thesis this chapter usually introduces the 
mathematical background, general theory that is the setting of your contribution.
\item In an empirical thesis it is usually chapter where you describe background of the problem, the 
methodology or explain the method(s) you will use to address the main problem 
of your thesis.
\end{itemize}

Reading other thesis of the same type will give you a good impression
of what is required of your own thesis.

\subsection{Main contribution}

The background part is followed by your contribution:
\begin{itemize}
\item In a fully mathematical (theoretical) thesis it is usually a sequence of
  definitions and lemmas of your own devising, which then culminate in
  the proof of your main theorem.
\item In an empirical thesis it is a set of empirical results obtained
  by applying a empirical research method.
\end{itemize}

\subsection{Experimental part}

In this part your results (theorems or new methods) are validated. This often means simulation, where your results are compared with other previous results.

You should present your contribution with precision, giving reasons
for the choices you have made.  You should follow the best practices
of the research tradition you are using.